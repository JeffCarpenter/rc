
%*******************************************************************************
%*                    START OF AUTOMATICALLY GENERATED TEX FILE                *
%*******************************************************************************
%*                                                                             *
%* This TeX file was automatically generated by the FunnelWeb preprocessor.    *
%* You can typeset this file to produce printed documentation by running it    *
%* through the TeX typesetter using a command such as:                         *
%*    tex thisfilename                                                         *
%* The resultant file thisfilename.dvi can be printed using a command such as: *
%*    lpr -Plaser -d thisfilename.dvi                                          *
%*                                                                             *
%* FunnelWeb is a preprocessor that allows programmers to weave programs and   *
%* their documentation together in a single document. The FunnelWeb program    *
%* analyses such documents producing both program files and typeset            *
%* documentation such as this TeX file.                                        *
%* FunnelWeb was created by Ross N. Williams.                                  *
%*                                                                             *
%* For more information on FunnelWeb see http://www.ross.net/funnelweb/        *
%*                                                                             *
%*******************************************************************************


%===================== Start of FunnelWeb TeX Definitions ======================


% Version
% -------
% This is FunnelWeb TeX Macro Library Version 1.0.


% Copyright
% ---------
% This set of FunnelWeb TeX definitions was written by Ross Williams and was
% originally Copyright (C) 1992 Ross N. Williams.  However, I, Ross Williams,
% hereby forego any claim to Copyright in this set of FunnelWeb TeX definitions
% and hereby authorize that the set of TeX definitions pass into the public
% domain. -- Ross N. Williams, 3:41pm 07-May-1992, Adelaide, Australia.


% Modification
% ------------
% Please record all modifications to these TeX definitions here. Unless
% otherwise specified, all modified definitions fall in the public domain too.
%
% Programmers:
%    RNW  Ross N. Williams  ross@ross.net
%
% Changes:
%    07-May-1992  RNW  Prepared this work for public domain release.
%    10-May-1999  RNW  Added \fwlib macro.


% General Comments
% ----------------
% This set of TeX definitions exists for two reasons:
%
%    1. To shorten and neaten the FunnelWeb TeX output.
%    2. To allow users to fiddle with the output format in their input files
%       (by inserting redefining "\def"s) without having to resort to
%       modifying the FunnelWeb code.
%
% The user is warned that these definitions may be changed from time to time
% (but probably not much). The user should not be too sneaky. In particular,
% users wishing to redefine some of these macros should do so in an explicitly
% defined section at the top of their input file. This will mean that in the
% event of problems, that section can simply be deleted or commented out to
% allow the document to at least be typeset in the default format. Users should
% limit themselves to redefining these macros in such a section and should
% refrain from using the macros throughout their documents.


% Environment Parameters
% ----------------------
% \tolerance tells TeX how tolerant it should be about making bad line and
% page breaks. Here we set it to it's maximum, as
%   1) Computer programs are likely to cause lots of bad breaks.
%   2) In most cases the user would probably rather get the TeX file through
%      TeX without any errors than fiddle with spacings for perfection.
\tolerance=10000

% I don't like indentation as it makes the page look more busy. Instead,
% paragraphs are separated by a little space (see next).
\parindent=0pt

% In many cases, users will produce documents with long runs of paragraphs.
% In order to space out these paragraphs, it is convenient to maintain a
% prevailing non-zero \parskip (end-of-paragaph skip). The only trouble is
% that the skip becomes a problem in macro definitions which require no skip
% and so we have to turn the skip on and off. The following two macros
% simplify this process.
\def\fwparskipon{\parskip=\medskipamount}
\def\fwparskipoff{\parskip=0pt}
\fwparskipon

% Setting raggedbottom allows TeX to leave a bit of space at the bottom of the
% page in order to better vertically align the rest of the page (e.g. skips
% won't stretch as much). It also means that headings are less likely to be
% isolated at the bottom of the page without any following text.
\raggedbottom


% Fonts
% -----
% Most of the typeset output is set in 10pt roman and 10pt tt font.
% The major extra font needs spring from titles and headings.
% For portability's sake we use only the following fonts:
%    cmr10
%    cmbx10
%    cmtt10
% and some enlargements of them. These fonts are all "standard" fonts
% in Plain TeX. See The TeXbook p.350.
\font\fwfontnote=cmr7

\font\fwfontnorm=cmr10
\font\fwfontnorma=cmr10 scaled \magstep1
\font\fwfontnormb=cmr10 scaled \magstep2

\font\fwfontbold=cmbx10
\font\fwfontbolda=cmbx10 scaled \magstep1
\font\fwfontboldb=cmbx10 scaled \magstep2
\font\fwfontboldc=cmbx10 scaled \magstep3
\font\fwfontboldd=cmbx10 scaled \magstep4


% Macros for Stylistic Details
% ----------------------------
% This section contains all the fiddly little macros for setting the details
% of each macro definition.

% Macro definitions are sandwiched by calls to these macros which can be used
% to sort out the spacing before and after the macro definition.
\def\fwbeginmacro{\fwparskipoff\bigskip}
\def\fwendmacro{\fwparskipon\par}

% These macros deal with the macro name and definition line.
\def\fwmacroname#1#2{{\sl #1\/}$\lbrack$#2$\rbrack$}
\def\fwfilename#1#2{{\bf #1}$\lbrack$#2$\rbrack$}
\def\fwzero#1{{\bf Z}}
\def\fwmany#1{{\bf M}}
\def\fwlib#1{{\bf L}}
\def\fwequals{ $\equiv$}
\def\fwplusequals{ $+\equiv$}

% Now for the actual body of the definition. It looks nice to have the tt
% code indented a little. Again, we use macros instead of writing direct TeX,
% so as to allow the user to fiddle this stuff to taste without having to
% modify the FunnelWeb C code.
\def\fwodef{\parindent=15pt\vskip0pt$\lbrace$\parindent=20pt}
\def\fwcdef{$\rbrace$\vskip0pt\parindent=0pt}
\def\fwoquote{`}
\def\fwcquote{'}
\def\fwoparen{$($}
\def\fwcomma{$,$}
\def\fwcparen{$)$}
\def\fwparam#1{$\diamond #1$}
\def\fwparams#1{$(\diamond #1)$}

% These macros deal with the notes that are appended at the end of each
% macro definition. Note that even though \fwisafile,\fwusedin, and \fwseealso
% have the same definition, they are given different names so as to allow the
% user to redefine these macros to typeset each kind of information differently
% if desired.
\def\fwbeginmacronotes{\begingroup\baselineskip=9pt\smallskip}
\def\fwnote#1{{\fwfontnote #1}\par}
\def\fwisafile#1{\fwnote{#1}}
\def\fwusedin#1{\fwnote{#1}}
\def\fwseealso#1{\fwnote{#1}}
\def\fwendmacronotes{\endgroup}


% Macros to Typeset Program Code Verbatim
% ---------------------------------------
% This is by far the hairiest and most difficult part of the typesetting task
% because we have to turn off most of TeX's natural instincts in order to
% typeset the program text exactly as it appears in the input file.
% Two macros are defined to pull this off: \fwbtx and \fwverbatimgobble.
% Their code was inspired by the following sections of "The TeXbook":
%    Appendix D: Dirty Tricks, 3.Verbatim listing, p.380-382.
%    Appendix E: Example Formats, p.421.
% The \fwbtx[ (for "FunnelWeb Begin TeXt") macro does most of the hard work.
% The liberal use of "%" is because I don't understand TeX well enough to
% understand when an end of line will cause trouble, and I am playing it safe.

% Before defining the main \fwbtx macro, we have to stash away some definitions
% in the hidden part of TeX's environment. Let's hope that these "hidden"
% definitions don't affect anything except what is desired to be affected.

% The tt font in which we wish to set the text has two Latin lurking ligatures!
% These are ?` and !`. To disable them, we define the left quote when ACTIVE
% to be defined in such a way as to prevent ligatures. The main TeX text will
% normally not be exposed to this definition because normally the leftquote
% character is not active. The \fwbtx macro temporarily makes the left quote
% character active thus activating the deactivation of left quote ligatures.
% See The TeXbook p.381.
{\catcode`\`=\active \gdef`{\relax\lq}}

% TeX is fairly carefree about spaces and so we have to make it more serious.
% To do so we pull the same trick as above, setting up a definition for active
% space, but only making space active during the span of the verbatim text.
% In Plain TeX the active space is defined to be simply a space, but here we
% define it to be a control space. This ensures that the space cannot
% be gobbled up by one of TeX's mysterious mechanisms when activated.
% See The TeXbook, p.381 and p.352.
{\obeyspaces\global\let =\ }

% Here is the main \fwbtx verbatim text macro.
% Note: The order in which all these pieces of business have to be done is
% still a partial mystery to me. Don't fiddle with this stuff unless you
% think you know what you are doing.
\def\fwbtx[{%
%
% The funnies involved in getting verbatim output are safely housed inside
% this \begingroup, and the \endgroup in \fwverbatimgobble. Groups are used
% instead of curly braces because we have to be able to signal the end of
% this macro with a curly brace.
\begingroup%
%
% \pars at the end of empty lines in the verbatim text won't come out normally
% because TeX is in vertical mode and they get gobbled up. To prevent this,
% we force \par to exit vertical mode first. See The TeXbook p.381.
\def\par{\leavevmode\endgraf}%
%
% Activate the leftquote character so as to avoid ligatures (see above).
\catcode`\`=\active%
%
% The \obeylines macro simply defines end of line (^M) to be \par. This ensures
% that TeX will treat each verbatim line as a new paragraph.
\obeylines%
%
% To get verbatim output, we have to desex all the special characters. This
% is explained in detail in The TeXbook p.380.
\def\do##1{\catcode`##1=12 }\dospecials%
%
% Activate the space character so as to make TeX treat blanks seriously.
% This activation invokes an eralier definition (see above).
\obeyspaces
%
% Interparagraph skips do not help the cause.
% Note: We have to preserve the indentation though, as the code is actually
% indented in the final output. See \fwodef in an earlier section.
\parskip=0pt%
%
% We typeset the verbatim text in tt font (courier on the Macintosh) for a
% number of reasons:
%    - tt font has the same horizontal spacing for each character.
%    - tt font covers the ASCII character set.
%    - tt font doesn't have many surprises (e.g. ligatures).
%    - tt font looks much what you might see on a computer terminal screen.
\tt%
%
% Having set up an environment for verbatim, we are ready to use it.
% By invoking \fwverbatimgobble, this \fwbtx macro gobbles up text verbatim (as
% part of the parameter of \fwverbatimgobble) until it sees the termination
% string "]fwetx=" (the "=" was thrown in to add obscurity as this sequence
% must never occur in the verbatim text).
\fwverbatimgobble}

% The \fwverbatimgobble macro exists to allow \fwbtx to bracket verbatim text.
\def\fwverbatimgobble#1]fwetx={#1\endgroup}


% Table of Contents
% -----------------
% The five levels of table of contents that FunnelWeb supports are identified
% by the five letters [A..E]. These are used throughout the following macros.

% The following macros are utilities to the TOC macros to follow.
\def\fwrule{\medskip\hrule\medskip}
\def\fwqh{\hskip1.5em\relax}
\def\fwbeforesec{\penalty-200\bigskip\medskip\par}

% The following macros are used to typeset the table of contents.
\def\fwtocstart#1{\fwrule\leftline{\fwfontbolda Table of Contents}\fwrule}
\def\fwtoca#1#2{\leftline{{\bf #1 #2}}}
\def\fwtocb#1#2{\leftline{\fwqh #1 #2}}
\def\fwtocc#1#2{\leftline{\fwqh\fwqh #1 #2}}
\def\fwtocd#1#2{\leftline{\fwqh\fwqh\fwqh #1 #2}}
\def\fwtoce#1#2{\leftline{\fwqh\fwqh\fwqh\fwqh #1 #2}}
\def\fwtocfinish#1{\fwrule}

% The following "library" macros define five different strengths of headings
% which can be used later in the section macros.
\def\fwliba#1#2{\vfill\eject{\fwfontboldc #1 #2}\penalty200\smallskip}
\def\fwlibb#1#2{\fwbeforesec{\fwfontboldb #1 #2}\penalty200\smallskip}
\def\fwlibc#1#2{\fwbeforesec{\fwfontnormb #1 #2}\penalty200\smallskip}
\def\fwlibd#1#2{\fwbeforesec{\bf          #1 #2}\penalty200}
\def\fwlibe#1#2{\fwbeforesec{\bf          #1 #2}}

% Here are the macros that actually typeset the section headings throughout
% the document. The fwlib system has been employed so as to easily allow the
% user to redefine the strengths of headings to taste. For example, the
% user could insert in the input document a similar set of definitions to these
% but with the b..e headings set to \fwlibc. This would tone down the output.
\def\fwseca#1#2{\fwliba{#1}{#2}}
\def\fwsecb#1#2{\fwlibb{#1}{#2}}
\def\fwsecc#1#2{\fwlibc{#1}{#2}}
\def\fwsecd#1#2{\fwlibd{#1}{#2}}
\def\fwsece#1#2{\fwlibe{#1}{#2}}


% Support for Explicit Typesetting
% --------------------------------
% FunnelWeb supports pragmas and other constructs that allow
% typesetter-independent typesetting commands to be given. The
% following macros support these features.

% The in-text literal @{sloth@} and emphasise @[walrus@] features.
\def\fwlit#1{{\tt #1}}
\def\fwemp#1{{\it #1}}

% The "@p new_page" pragma.
\def\fwnewpage{\vfill\eject}

% The "@p vskip Nmm" pragma.
\def\fwvskip#1{\null\vskip #1mm}

% The "@p title <font> <align> <text>" pragma.
\def\fwfontnormal#1{{\fwfontnorm {#1}}}
\def\fwfonttitle#1{{\fwfontboldd {#1}}}
\def\fwfontsmalltitle#1{{\fwfontboldb {#1}}}
\def\fwleftline#1{\leftline{#1}}
\def\fwcenterline#1{\centerline{#1}}
\def\fwrightline#1{\rightline{#1}}


% Support for Old FunnelWeb
% -------------------------
% The following macros were used extensively in the first version of
% FunnelWeb and are retained so that these older input files will still
% typeset cleanly.
\def\p#1{{\tt #1}}  % P for Program text.
\def\flagpage#1#2{
   \null
   \vfill
   \centerline{\fwfontboldd #1}
   \vskip 1cm
   \centerline{\fwfontboldd #2}
   \vfill
   \null
   \vfill
}

%====================== End of FunnelWeb TeX Definitions =======================



\documentstyle{report}

\title{
  An Emacs/Elisp Major Mode for FunnelWeb
}

\author{
  A.B.Coates \\
  Department of Physics \\
  The University of Queensland \\
  QLD\ \ 4072 \\
  Australia \\
  Email: {\tt coates@physics.uq.oz.au}
}

\begin{document}

\maketitle

\tableofcontents

\chapter{Major and Minor Modes}

\section{The FunnelWeb Major Mode}
This is the definition of the function which defines the FunnelWeb
mode.  A brief (and not exhaustive) description of the functions
available is given in its description string.

\fwbeginmacro
\fwmacroname{fw-mode Function Definitions}{1}\fwplusequals \fwodef \fwbtx[
(defun fw-mode ()
  "Major mode for editing FunnelWeb files.  Built on top of
tex-mode.  Makes @} , @) , and @> display their matching
opening braces @{ , @( , or @< .

Use \\[fw-buffer] to run FunnelWeb on the current buffer
*without* saving it.
Use \\[fw-file] to be prompted to save the buffer to a file
first, before running FunnelWeb on the file.
Use \\[fw-tex] to run the (La)TeX output file through TeX or
LaTeX as appropriate.
Use \\[fw-print] to print a .dvi file.
Use \\[fw-show-print-queue] to show the print queue that
\\[fw-print] put your job on.
Commands from tex-mode (or a similar selected mode) are
mapped to operate on the (La)TeX file produced by FunnelWeb.

Key sequences:

see documentation for `switch-mode' for key sequences
relating to mode switching.

Mode variables:
fw-autofill-mode
        Whether autofill-mode is automatically invoked for
        FunnelWeb files.  Set to 0 for no, to a positive
        number for yes.
fw-TeX-mode
        The major (La)TeX mode on which fw-mode is based.
        Typically either 'tex-mode or 'latex-mode
        (or perhaps 'tex-init for Auc-TeX
         or 'html-mode for hypertext ...).
fw-directory
        Directory in which to create temporary files for
        TeX jobs run by \\[fw-buffer] or \\[fw-file].
fw-use-TeX-quote-style
        Set this to nil so the the \" key produces the
        normal \" character.  If set to true, the \" key
        works as for (La)TeX mode, producing either
        `` or ''.
fw-dvi-print-command
        Command string used by \\[fw-print] to print a
        .dvi file.
fw-show-queue-command
        Command string used by \\[fw-show-print-queue] to
        show the print queue that \\[fw-print] put your
        job on.

Entering FunnelWeb-mode calls the value of fw-mode-hook."
  (interactive)
  (funcall fw-TeX-mode)
  (hack-local-variables)
  (setq mode-name "FunnelWeb")
  (setq major-mode 'fw-mode)
  (auto-fill-mode fw-auto-fill-mode)
  (if fw-mode-map (use-local-map fw-mode-map))
  (setq fw-buffer-p t)
  (switch-mode fw-switch-minor-mode-init)
  (run-hooks 'fw-mode-hook))
]fwetx=%
\fwcdef 
\fwbeginmacronotes
\fwseealso{This macro is defined in definitions 1 and 9.}
\fwusedin{This macro is invoked in definition 37.}
\fwendmacronotes
\fwendmacro


\section{The Switch Minor Mode}

This is the definition of the function which defines the Switch
minor mode.  A brief (and not exhaustive) description of the functions
available is given in its description string.

\fwbeginmacro
\fwmacroname{switch-mode Function Definitions 2}{2}\fwplusequals \fwodef \fwbtx[
(defun switch-mode
  ( &optional set-mode-p
              no-check-p
              region-open-string
              region-close-string
              region-boundary-pattern )
  "With no arguments, toggle whether the region mode
checking is activated or not.  Optional first argument
SET-MODE-P can be used to definitely switch the region
mode checking on or off: a positive number for on, off
otherwise.
    Optional second argument NO-CHECK-P is nil for a
check of whether the point is in a code region or not
to be taken immediately.  With any other value, no check
is done.
    Optional third and fourth arguments REGION-OPEN-STRING
and REGION-CLOSE-STRING define the boundaries by which a
code region is recognised.
    Optional fourth argument REGION-BOUNDARY_PATTERN should
be a pattern equivalent to
    \\(REGION-OPEN-STRING\\|REGION-CLOSE-STRING\\)
i.e. a pattern which recognises either an opening or closing
pattern.

Key sequences:

\\C-x\\C-a key sequences are used for the \"Switch\" minor
mode, which allows modes to be swapped while editing a file.

\\C-x \\C-a t : toggle Switch minor-mode on and off
\\C-x \\C-a c : check whether in macro region or not
\\C-x \\C-a p : toggles whether the default macro region
                mode is used with or without prompting
\\C-x \\C-a s : set the current macro region mode"
  (interactive)
]fwetx=%
\fwcdef 
\fwbeginmacronotes
\fwseealso{This macro is defined in definitions 2, 4, 6, 14, 15, 16, 18, 19, 20, 21, 23, 26 and 30.}
\fwusedin{This macro is invoked in definition 32.}
\fwendmacronotes
\fwendmacro


A variable, local to each buffer, is needed to note whether
switch-mode is active or not.

\fwbeginmacro
\fwmacroname{switch-mode Variable Definitions 1}{3}\fwplusequals \fwodef \fwbtx[
(defvar switch-mode nil
  "Whether the region checking minor mode is active or not.")
(make-variable-buffer-local 'switch-mode)
]fwetx=%
\fwcdef 
\fwbeginmacronotes
\fwseealso{This macro is defined in definitions 3, 5, 13, 17 and 29.}
\fwusedin{This macro is invoked in definition 32.}
\fwendmacronotes
\fwendmacro


The boolean value {\tt switch-on-p} is given the value {\tt t} if the
switch minor mode should be turned on, {\tt nil} otherwise.  The value
depends on when the function call toggles the state or sets it
directly.

\fwbeginmacro
\fwmacroname{switch-mode Function Definitions 2}{4}\fwplusequals \fwodef \fwbtx[  (let ((switch-on-p
         (or (and (numberp set-mode-p)
                  (> set-mode-p 0))
             (not switch-mode))))
    (if switch-on-p
        (let ((old-switch-mode-value
               switch-mode))
          (setq switch-mode t
                switch-force-region-check-p t)
          (if (not (or no-check-p old-switch-mode-value))
              (switch-check-if-in-region)))
      (setq switch-mode nil))
]fwetx=%
\fwcdef 
\fwbeginmacronotes
\fwseealso{This macro is defined in definitions 2, 4, 6, 14, 15, 16, 18, 19, 20, 21, 23, 26 and 30.}
\fwusedin{This macro is invoked in definition 32.}
\fwendmacronotes
\fwendmacro


Three variables, with local values in each buffer, are required to
hold the values of the region opening and closing strings and a
combined pattern.

\fwbeginmacro
\fwmacroname{switch-mode Variable Definitions 1}{5}\fwplusequals \fwodef \fwbtx[
(defvar switch-open-region-string "@{"
  "String defining the beginning of a code region.")
(make-variable-buffer-local 'switch-open-region-string)

(defvar switch-close-region-string "@}"
  "String defining the end of a code region.")
(make-variable-buffer-local 'switch-close-region-string)

(defvar switch-boundary-region-pattern "@\\({\\|}\\)"
  "Regular expression defining the beginning or end of a code
region.")
(make-variable-buffer-local 'switch-boundary-region-pattern)
]fwetx=%
\fwcdef 
\fwbeginmacronotes
\fwseealso{This macro is defined in definitions 3, 5, 13, 17 and 29.}
\fwusedin{This macro is invoked in definition 32.}
\fwendmacronotes
\fwendmacro


\fwbeginmacro
\fwmacroname{switch-mode Function Definitions 2}{6}\fwplusequals \fwodef \fwbtx[    (if region-open-string
        (setq switch-open-region-string
              region-open-string))
    (if region-close-string
        (setq switch-close-region-string
              region-close-string))
    (if region-boundary-pattern
        (setq switch-boundary-region-pattern
              region-boundary-pattern))
    (force-mode-line-update t)))
]fwetx=%
\fwcdef 
\fwbeginmacronotes
\fwseealso{This macro is defined in definitions 2, 4, 6, 14, 15, 16, 18, 19, 20, 21, 23, 26 and 30.}
\fwusedin{This macro is invoked in definition 32.}
\fwendmacronotes
\fwendmacro


\section{Invocation of the FunnelWeb mode}
To cause FunnelWeb to be invoked for {\tt .fw} and {\tt .fwi} files,
the following lines should be added to the user's ``{\tt .emacs}''
file.

\fwbeginmacro
\fwmacroname{Emacs initialisation file additions}{7}\fwzero{}\fwequals \fwodef \fwbtx[
(autoload 'fw-mode "fw-mode")
(setq
 auto-mode-alist
 (reverse
  (append
   (reverse auto-mode-alist)
   (list
    (append
     (list "\\.fw$")
     'fw-mode
     )
    (append
     (list "\\.fwi$")
     'fw-mode
     )
    )
   )
  )
 )
]fwetx=%
\fwcdef 
\fwbeginmacronotes
\fwusedin{This macro is NEVER invoked.}
\fwendmacronotes
\fwendmacro


\chapter{Function and Variable Definitions for the FunnelWeb Mode}
An alternative command for saving buffers converts all tabs to spaces
before saving, since FunnelWeb does not accept tab characters in .fw
files by default.  Because mode-switching is expected to be used when
editing FunnelWeb files, a local variable in each buffer is set to
define whether a buffer is a FunnelWeb buffer or not.  A global
variable is also provided to switch off the replacement of tabs by
spaces.  The string used to replace tabs with spaces can also be
changed from the default of 8 spaces.  Another global variable also
controls the removal of trailing spaces, since these are also not
accepted by FunnelWeb by default.

\fwbeginmacro
\fwmacroname{fw-mode Variable Definitions2}{8}\fwplusequals \fwodef \fwbtx[
(defvar fw-untabify-before-saving-p t
  "*Whether to convert tabs to spaces before saving FunnelWeb
buffers.")

(defvar fw-tab-string "	"
  "Tab character as string.")

(defvar fw-tab-space-replacement "        "
  "*String of spaces for replacing tabs.")
(make-variable-buffer-local 'fw-tab-space-replacement)

(defvar fw-remove-trailing-spaces-before-saving-p t
  "*Whether to remove trailing spaces on lines before saving
FunnelWeb buffers.")

(defvar fw-space (string-to-char " ")
  "Space character, used in removing trailing spaces.")

(defvar fw-trailing-space-pattern " +$"
  "*Pattern used to identify trailing spaces on lines.")

(defvar fw-null-string ""
  "Empty string used for replacing trailing spaces.")
]fwetx=%
\fwcdef 
\fwbeginmacronotes
\fwseealso{This macro is defined in definitions 8 and 11.}
\fwusedin{This macro is invoked in definition 36.}
\fwendmacronotes
\fwendmacro


\fwbeginmacro
\fwmacroname{fw-mode Function Definitions}{9}\fwplusequals \fwodef \fwbtx[
(defun fw-untabify-save-buffer ()
  "Save the buffer, optionally removing tabs and trailing spaces
first if the buffer is a FunnelWeb buffer."
  (interactive)
  (if fw-buffer-p
      (progn
        (if fw-untabify-before-saving-p
            (progn
              (message "Removing tabs ...")
              (save-excursion
                (goto-char (point-min))
                (while (search-forward fw-tab-string nil t)
                  (replace-match fw-tab-space-replacement nil t)))))
        (if fw-remove-trailing-spaces-before-saving-p
            (progn
              (message "Removing trailing spaces ...")
              (save-excursion
                (goto-char (point-min))
                (while (< (point) (point-max))
                  (end-of-line)
                  (while (eq (preceding-char) fw-space)
                    (backward-delete-char 1))
                  (forward-line 1)))))))
  (save-buffer))
]fwetx=%
\fwcdef 
\fwbeginmacronotes
\fwseealso{This macro is defined in definitions 1 and 9.}
\fwusedin{This macro is invoked in definition 37.}
\fwendmacronotes
\fwendmacro


Other variable definitions follows.  Those for which the decription
text begins with a \verb$*$ can be changed on the command line by the
user, and all can be set in the users {\tt .emacs} file.

\fwbeginmacro
\fwmacroname{fw-mode Variable Definitions1}{10}\fwplusequals \fwodef \fwbtx[
(defvar fw-TeX-mode 'tex-mode
  "*The default TeX-based foundation for FunnelWeb mode.")
]fwetx=%
\fwcdef 
\fwbeginmacronotes
\fwseealso{This macro is defined in definitions 10.}
\fwusedin{This macro is invoked in definition 36.}
\fwendmacronotes
\fwendmacro


\fwbeginmacro
\fwmacroname{fw-mode Variable Definitions2}{11}\fwplusequals \fwodef \fwbtx[
(defvar fw-switch-minor-mode-init 1
  "*Whether to activate the Switch minor mode automatically
on entering FunnelWeb mode.  A positive integer for yes.")

(defvar fw-buffer-p nil
  "Whether a buffer is a FunnelWeb buffer or not.")
(make-variable-buffer-local 'fw-buffer-p)
(switch-add-to-preservation-list 'fw-buffer-p)

(defvar fw-quote-style t
  "*Whether to map \" to a single character or use (La)TeX
mapping to `` or '' as appropriate.  Can be set to nil to
remove the (La)TeX mapping, or anything else to enable it.")

(defvar fw-auto-fill-mode 1
  "*Whether autofill-mode is automatically invoked for
FunnelWeb files.  Set to 0 for no, to a positive number
for yes.")

(defvar fw-command "fw"
  "*The command to run FunnelWeb on a file.
Any pre-options (fw-command-pre-options) will be appended
to this string, separated by a space, followed by the
filename, also separated by a space, and finally any
post-options (fw-command-post-options), again separated
by a space.")

(defvar fw-command-pre-options nil
  "*Options which go before the filename when
calling FunnelWeb.")
(make-variable-buffer-local 'fw-command-pre-options)

(defvar fw-command-post-options "+t +D"
  "*Options which go after the filename when
calling FunnelWeb.")
(make-variable-buffer-local 'fw-command-post-options)

(defvar fw-shell-cd-command "cd"
  "*Command to give to shell running FunnelWeb to
change directory.  The value of fw-directory will be
appended to this, separated by a space.")

(defvar fw-mode-syntax-table nil
  "Syntax table used while in FunnelWeb mode.")

(defvar fw-mode-map nil
  "Keymap for FunnelWeb mode.")

(defvar fw-close-definition-block-additive t
  "*Whether definitions should be closed using \"+=\" (t) or
\"==\" (nil).")
(make-variable-buffer-local
 'fw-close-definition-block-additive)

(defvar fw-close-definition-block-newline-suppress t
  "*Whether definitions should be closed using \"@-\".")
(make-variable-buffer-local
 'fw-close-definition-block-newline-suppress)

(defvar fw-close-definition-block-blank-line t
  "Whether definitions should be closed leaving
a blank line.")
(make-variable-buffer-local
 'fw-close-definition-block-blank-line)
]fwetx=%
\fwcdef 
\fwbeginmacronotes
\fwseealso{This macro is defined in definitions 8 and 11.}
\fwusedin{This macro is invoked in definition 36.}
\fwendmacronotes
\fwendmacro


\chapter{Key Definitions for the FunnelWeb Mode}
These are the special key definitions for FunnelWeb commands.

\fwbeginmacro
\fwmacroname{fw-mode Key Definitions}{12}\fwequals \fwodef \fwbtx[
(global-set-key "\C-x\C-s" 'fw-untabify-save-buffer)
]fwetx=%
\fwcdef 
\fwbeginmacronotes
\fwusedin{This macro is invoked in definition 38.}
\fwendmacronotes
\fwendmacro


\chapter{Function and Variable Definitions for the Switch Minor Mode}
The Switch minor mode works by adding a function, run after each emacs
command, which checks the position of point.  If point is outside of
the current region, each code or text, a function is run to determine
whether point lies in a code or text region.  Note that for speed, the
code does not update the boundaries of the current region after each
command, and so the checking may be fooled when and if sections are
deleted from a code or text region.

Variables are needed to hold the bounds of the current region and the
whether flag marking whether a region check is in progress or not (to
avoid an infinite recursion).

\fwbeginmacro
\fwmacroname{switch-mode Variable Definitions 1}{13}\fwplusequals \fwodef \fwbtx[
(defvar switch-region-min nil
  "Start of current macro or text region, or nil.")
(make-variable-buffer-local 'switch-region-min)

(defvar switch-region-max nil
  "End of current macro or text region, or nil.")
(make-variable-buffer-local 'switch-region-max)

(defvar switch-not-in-region-check-routine-p t
  "Whether a region check is not in progress.")
(make-variable-buffer-local
 'switch-not-in-region-check-routine-p)
]fwetx=%
\fwcdef 
\fwbeginmacronotes
\fwseealso{This macro is defined in definitions 3, 5, 13, 17 and 29.}
\fwusedin{This macro is invoked in definition 32.}
\fwendmacronotes
\fwendmacro


\fwbeginmacro
\fwmacroname{switch-mode Function Definitions 2}{14}\fwplusequals \fwodef \fwbtx[
(defun switch-post-command-hook-function ()
  "Function to check if the user is in a code definition
region or not."
  (if (and (not isearch-mode)
           switch-mode
           switch-not-in-region-check-routine-p
           (or (not switch-region-min)
               (not switch-region-max)
               (let ((the-point (point)))
                 (or
                  (< the-point switch-region-min)
                  (> the-point switch-region-max)))))
      (let ((switch-not-in-region-check-routine-p nil))
            (switch-check-if-in-region))))
]fwetx=%
\fwcdef 
\fwbeginmacronotes
\fwseealso{This macro is defined in definitions 2, 4, 6, 14, 15, 16, 18, 19, 20, 21, 23, 26 and 30.}
\fwusedin{This macro is invoked in definition 32.}
\fwendmacronotes
\fwendmacro


The function for checking whether the point is in a code or text
region or not is as follows.

\fwbeginmacro
\fwmacroname{switch-mode Function Definitions 2}{15}\fwplusequals \fwodef \fwbtx[
(defun switch-check-if-in-region ()
  "Check if point is inside or outside a FunnelWeb macro
definition. If outside, switch to (Auc/La)TeX mode.
If inside, either ask the user for a mode, defaulting
to the last mode used in a macro definition, or directly
use the last mode chosen (depends on value of variable
switch-mode-prompt-p)."
  (interactive)
]fwetx=%
\fwcdef 
\fwbeginmacronotes
\fwseealso{This macro is defined in definitions 2, 4, 6, 14, 15, 16, 18, 19, 20, 21, 23, 26 and 30.}
\fwusedin{This macro is invoked in definition 32.}
\fwendmacronotes
\fwendmacro


First, searches are carried out forwards and backwards from point to
determine what sort of region point is in.

\fwbeginmacro
\fwmacroname{switch-mode Function Definitions 2}{16}\fwplusequals \fwodef \fwbtx[  (let ((backward-search-result
        (switch-search-backward-check
         switch-boundary-region-pattern
         switch-open-region-string))
        (forward-search-result
        (switch-search-forward-check
         switch-boundary-region-pattern
         switch-close-region-string))
        (backward-search-value)
        (forward-search-value))
    (setq backward-search-value
          (car backward-search-result))
    (setq forward-search-value
          (car forward-search-result))
    (setq switch-region-min
          (car (cdr backward-search-result)))
    (setq switch-region-max
          (car (cdr forward-search-result)))
    (if (not (or (< backward-search-value 0)
                 (< forward-search-value 0)
                 (and (eq backward-search-value 0)
                      (eq forward-search-value 0))))
]fwetx=%
\fwcdef 
\fwbeginmacronotes
\fwseealso{This macro is defined in definitions 2, 4, 6, 14, 15, 16, 18, 19, 20, 21, 23, 26 and 30.}
\fwusedin{This macro is invoked in definition 32.}
\fwendmacronotes
\fwendmacro


The following code determines what is done if point is inside a code
region.  Variables are required to define the behaviour, and to store
the current code region mode, as well as to store whether point was
last in a code or text region.

\fwbeginmacro
\fwmacroname{switch-mode Variable Definitions 1}{17}\fwplusequals \fwodef \fwbtx[
(defvar switch-force-region-check-p nil
  "Whether a region check should be forced or not,
regardless of whether it seems to be required or not.")
(make-variable-buffer-local 'switch-force-region-check-p)

(defvar switch-mode-prompt-p t
  "*Whether the user should be prompted for a new
major mode type each time point enters a code region.")
(make-variable-buffer-local 'switch-mode-prompt-p)

(defvar switch-current-mode nil
  "The current mode for a code region.")
(make-variable-buffer-local 'switch-current-mode)

(defvar switch-currently-in-region-p nil
  "Whether point is in a code region or not.")
(make-variable-buffer-local 'switch-currently-in-region-p)

(defvar switch-highlight-p nil
  "Whether to highlight regions as they are entered.
(This requires the `hilit19' library to be loaded.)")
(make-variable-buffer-local 'switch-highlight-p)
]fwetx=%
\fwcdef 
\fwbeginmacronotes
\fwseealso{This macro is defined in definitions 3, 5, 13, 17 and 29.}
\fwusedin{This macro is invoked in definition 32.}
\fwendmacronotes
\fwendmacro


\fwbeginmacro
\fwmacroname{switch-mode Function Definitions 2}{18}\fwplusequals \fwodef \fwbtx[        (if (or (not switch-currently-in-region-p)
                switch-force-region-check-p)
            (progn
              (switch-select-mode (not switch-mode-prompt-p))
              (if switch-current-mode
                  (progn
                    (message (concat "Changed to "
                                     (prin1-to-string
                                      switch-current-mode)))
                    (setq switch-currently-in-region-p t)))))
]fwetx=%
\fwcdef 
\fwbeginmacronotes
\fwseealso{This macro is defined in definitions 2, 4, 6, 14, 15, 16, 18, 19, 20, 21, 23, 26 and 30.}
\fwusedin{This macro is invoked in definition 32.}
\fwendmacronotes
\fwendmacro


Now the code for the case where point is {\em not} in a code region,
but in a text region.

\fwbeginmacro
\fwmacroname{switch-mode Function Definitions 2}{19}\fwplusequals \fwodef \fwbtx[      (if switch-currently-in-region-p
          (let ((preservation-list
                 (switch-get-preservation-values)))
            (fw-mode)
            (switch-set-preservation-values
             preservation-list)
            (message "Changed to fw-mode")
            (setq switch-currently-in-region-p nil))))
    (setq switch-force-region-check-p nil)))
]fwetx=%
\fwcdef 
\fwbeginmacronotes
\fwseealso{This macro is defined in definitions 2, 4, 6, 14, 15, 16, 18, 19, 20, 21, 23, 26 and 30.}
\fwusedin{This macro is invoked in definition 32.}
\fwendmacronotes
\fwendmacro


A function is provided to allow the user to toggle whether the mode is
prompted for each time a code region is entered or not.

\fwbeginmacro
\fwmacroname{switch-mode Function Definitions 2}{20}\fwplusequals \fwodef \fwbtx[
(defun switch-mode-prompt-toggle ()
  "Toggle whether the user is prompted for the major mode
each time a code region is entered."
  (interactive)
  (setq switch-mode-prompt-p (not switch-mode-prompt-p))
  (if switch-mode-prompt-p
      (message "Mode prompting switched on.")
    (message "Mode prompting switched off.")))
]fwetx=%
\fwcdef 
\fwbeginmacronotes
\fwseealso{This macro is defined in definitions 2, 4, 6, 14, 15, 16, 18, 19, 20, 21, 23, 26 and 30.}
\fwusedin{This macro is invoked in definition 32.}
\fwendmacronotes
\fwendmacro


It is also sometimes useful to be able to force a major mode to be
asked for.  This is necessary when editing changes cause the program
to mistake the boundaries of the current region.

\fwbeginmacro
\fwmacroname{switch-mode Function Definitions 2}{21}\fwplusequals \fwodef \fwbtx[
(defun switch-mode-force-prompt ()
  "Cause the user to be prompted for a mode type for
the current region, if a code region."
  (interactive)
  (let ((switch-force-region-check-p t))
    (switch-check-if-in-region)))
]fwetx=%
\fwcdef 
\fwbeginmacronotes
\fwseealso{This macro is defined in definitions 2, 4, 6, 14, 15, 16, 18, 19, 20, 21, 23, 26 and 30.}
\fwusedin{This macro is invoked in definition 32.}
\fwendmacronotes
\fwendmacro


The mode selection function allows the user to choose from any command
which ends with string \"-mode\".  It optionally beeps on activation,
to note to the user that a mode selection is necessary.

\fwbeginmacro
\fwmacroname{switch-mode Variable Definitions 2}{22}\fwplusequals \fwodef \fwbtx[
(defvar switch-mode-list (make-switch-mode-list)
  "Return a list of strings of all possible major modes
from which the user can choose.")

(defvar switch-beep-on-mode-selection-p t
  "*Whether to beep when a mode selection is necessary.")
]fwetx=%
\fwcdef 
\fwbeginmacronotes
\fwseealso{This macro is defined in definitions 22.}
\fwusedin{This macro is invoked in definition 32.}
\fwendmacronotes
\fwendmacro


\fwbeginmacro
\fwmacroname{switch-mode Function Definitions 2}{23}\fwplusequals \fwodef \fwbtx[
(defun switch-select-mode ( &optional no-prompt-p )
  "Interactively get the user to select a macro mode,
giving the last-used macro mode as a default, and allowing
the user to select from all possible major mode commands
with name completion.
    Optional parameter NO-PROMPT-P, if non-nil, stops
prompting from taking place unless the current mode
(switch-current-mode) is nil."
  (let ((old-switch-mode-value
         (if switch-mode 1 0)))
    (if (not (and no-prompt-p switch-current-mode))
        (progn
          (if switch-beep-on-mode-selection-p (beep))
          (let ((new-mode (completing-read
                           "Mode: "
                           switch-mode-list
                           nil
                           nil
                           (switch-convert-to-default-string
                            switch-current-mode)
                           nil)))
            (setq switch-current-mode
                  (car (read-from-string new-mode))))))
    (let ((preservation-list
           (switch-get-preservation-values))
          (region-min switch-region-min)
          (region-max switch-region-max))
      (message "Changing to %s"
               (prin1-to-string switch-current-mode))
      (if switch-current-mode (funcall switch-current-mode))
      (if (and switch-highlight-p region-min region-max)
          (progn
            (hilit-unhighlight-region region-min region-max)
            (hilit-highlight-region region-min region-max)))
      (switch-set-preservation-values preservation-list))
    (switch-mode old-switch-mode-value t)
    (force-mode-line-update)))
]fwetx=%
\fwcdef 
\fwbeginmacronotes
\fwseealso{This macro is defined in definitions 2, 4, 6, 14, 15, 16, 18, 19, 20, 21, 23, 26 and 30.}
\fwusedin{This macro is invoked in definition 32.}
\fwendmacronotes
\fwendmacro


A short function to convert an object to a string if non-nil.

\fwbeginmacro
\fwmacroname{switch-mode Function Definitions 1}{24}\fwplusequals \fwodef \fwbtx[
(defun switch-convert-to-default-string ( object )
  "Convert an object into a default string for
name completion."
  (if object (prin1-to-string object)))
]fwetx=%
\fwcdef 
\fwbeginmacronotes
\fwseealso{This macro is defined in definitions 24, 25, 27 and 28.}
\fwusedin{This macro is invoked in definition 32.}
\fwendmacronotes
\fwendmacro


The following function returns a list of possible modes, in the form
of an Emacs ``alist''.  A second function allows the user to update
the list.

\fwbeginmacro
\fwmacroname{switch-mode Function Definitions 1}{25}\fwplusequals \fwodef \fwbtx[
(defun make-switch-mode-list ()
  "Return a list of all commands ending in \"-mode\",
in the form of an alist."
  (let ((mode-list (apropos-internal "-mode$")))
    (mapcar 'switch-convert-to-alist-string mode-list)))
]fwetx=%
\fwcdef 
\fwbeginmacronotes
\fwseealso{This macro is defined in definitions 24, 25, 27 and 28.}
\fwusedin{This macro is invoked in definition 32.}
\fwendmacronotes
\fwendmacro


\fwbeginmacro
\fwmacroname{switch-mode Function Definitions 2}{26}\fwplusequals \fwodef \fwbtx[
(defun update-switch-mode-list ()
  "Update the mode list used for the switch minor-mode."
  (interactive)
  (setq switch-mode-list (make-switch-mode-list)))
]fwetx=%
\fwcdef 
\fwbeginmacronotes
\fwseealso{This macro is defined in definitions 2, 4, 6, 14, 15, 16, 18, 19, 20, 21, 23, 26 and 30.}
\fwusedin{This macro is invoked in definition 32.}
\fwendmacronotes
\fwendmacro


This simple function converts objects to a form suitable for an Elisp
``alist''.

\fwbeginmacro
\fwmacroname{switch-mode Function Definitions 1}{27}\fwplusequals \fwodef \fwbtx[
(defun switch-convert-to-alist-string ( object )
  "Convert an object to a string in its own list,
the format appropriate for an alist."
  (list (prin1-to-string object)))
]fwetx=%
\fwcdef 
\fwbeginmacronotes
\fwseealso{This macro is defined in definitions 24, 25, 27 and 28.}
\fwusedin{This macro is invoked in definition 32.}
\fwendmacronotes
\fwendmacro


Two functions are used to search for code region boundaries, one for
searching forwards from point, the other for searching backwards.
Each searches for the given regexp, and if found, compares it to the
given string.  A list is returned containing an integer value
describing the search result and the position of the string which was
matched.

\fwbeginmacro
\fwmacroname{switch-mode Function Definitions 1}{28}\fwplusequals \fwodef \fwbtx[
(defun switch-search-forward-check ( regexp string )
  "Search forward for the given regexp, and if the matching
pattern is found and it matches the given string, return `1'
and the beginning of the pattern, else return `-1' and the
beginning of the pattern.  If the pattern is not found when
the end of the buffer is reached, return `0' and the end of
the buffer."
  (save-excursion
    (if (search-forward-regexp regexp (point-max) t)
        (if (string-equal
             string
             (buffer-substring
              (match-beginning 0)
              (match-end 0)))
            (list 1 (match-beginning 0))
          (list -1 (match-beginning 0)))
      (list 0 (point-max)))))

(defun switch-search-backward-check ( regexp string )
  "Search backward for the given regexp, and if the matching
pattern is found and it matches the given string, return `1'
and the end of the pattern, else return `-1' and the end of
the pattern.  If the pattern is not found when the beginning
of the buffer is reached, return `0' and the beginning of the
buffer."
  (save-excursion
    (if (search-backward-regexp regexp (point-min) t)
        (if (string-equal
             string
             (buffer-substring
              (match-beginning 0)
              (match-end 0)))
            (list 1 (match-end 0))
          (list -1 (match-end 0)))
      (list 0 (point-min)))))
]fwetx=%
\fwcdef 
\fwbeginmacronotes
\fwseealso{This macro is defined in definitions 24, 25, 27 and 28.}
\fwusedin{This macro is invoked in definition 32.}
\fwendmacronotes
\fwendmacro


It is useful to allow functions to maintain a list of variables which
should be preserved by switch-mode operations.  This list can be
changed by the user.

\fwbeginmacro
\fwmacroname{switch-mode Variable Definitions 1}{29}\fwplusequals \fwodef \fwbtx[
(defvar switch-preservation-list
  (list 'switch-current-mode 'switch-mode-prompt-p)
  "List of names of variables that need to be specially
preserved by switch-mode operations.")
]fwetx=%
\fwcdef 
\fwbeginmacronotes
\fwseealso{This macro is defined in definitions 3, 5, 13, 17 and 29.}
\fwusedin{This macro is invoked in definition 32.}
\fwendmacronotes
\fwendmacro


\fwbeginmacro
\fwmacroname{switch-mode Function Definitions 2}{30}\fwplusequals \fwodef \fwbtx[
(defun switch-add-to-preservation-list ( var )
  "Add a variable to the switch-mode preservation list."
  (if (not (memq var switch-preservation-list))
      (setq switch-preservation-list
            (append switch-preservation-list (list var)))))

(defun switch-get-preservation-values ()
  "Returns a list of current values which can be used by
switch-set-preservation-values to restore the values of
the variables in the preservation list
(switch-preservation-list)."
  (if switch-preservation-list
      (mapcar
       'switch-preserve-form
       switch-preservation-list)))

(defun switch-preserve-form ( var )
  "Return a list of the variable name and its value."
  (list var (eval var)))

(defun switch-set-preservation-values ( vallist )
  "Reset the variables in the preservation value list
using the value list previously created by
switch-get-preservation-values."
  (mapcar 'switch-apply-set vallist))

(defun switch-apply-set ( varpair )
  "Given a list of the form (NAME . VALUE), assign the value
to the name."
  (apply 'set varpair))
]fwetx=%
\fwcdef 
\fwbeginmacronotes
\fwseealso{This macro is defined in definitions 2, 4, 6, 14, 15, 16, 18, 19, 20, 21, 23, 26 and 30.}
\fwusedin{This macro is invoked in definition 32.}
\fwendmacronotes
\fwendmacro


\chapter{Key Definitions for the Switch Minor Mode}
These are the special key definitions for Switch minor-mode commands.

\fwbeginmacro
\fwmacroname{switch-mode Key Definitions}{31}\fwequals \fwodef \fwbtx[
(defvar switch-prefix nil
  "Switch minor-mode region check \\C-x\\C-a keymap.")

(define-prefix-command 'switch-prefix)

(global-set-key "\C-x\C-a" 'switch-prefix)

(define-key switch-prefix "c" 'switch-check-if-in-region)

(define-key switch-prefix "t" 'switch-mode)

(define-key switch-prefix "p" 'switch-mode-prompt-toggle)

(define-key switch-prefix "s" 'switch-mode-force-prompt)
]fwetx=%
\fwcdef 
\fwbeginmacronotes
\fwusedin{This macro is invoked in definition 32.}
\fwendmacronotes
\fwendmacro


\chapter{Construction of the elisp files}
The file ``{\tt switch-mode.el}'' requires the standard Emacs
distribution provides the new minor mode function `{\tt switch-mode}'.

The variable and function definitions are each divided up into two
lots, in the order
\begin{itemize}
  \item variables which don't rely on functions for their default
        values
  \item functions which don't rely on variables in their definitions
  \item variables which rely on functions for their default
        values
  \item functions which rely on variables in their definitions.
\end{itemize}

\fwbeginmacro
\fwmacroname{switch-mode Constructed Code}{32}\fwplusequals \fwodef \fwbtx[(provide 'switch-mode)

]fwetx=%
\fwmacroname{switch-mode Variable Definitions 1}{3}\fwbtx[

]fwetx=%
\fwmacroname{switch-mode Function Definitions 1}{24}\fwbtx[

]fwetx=%
\fwmacroname{switch-mode Variable Definitions 2}{22}\fwbtx[

]fwetx=%
\fwmacroname{switch-mode Function Definitions 2}{2}\fwbtx[

]fwetx=%
\fwmacroname{switch-mode Key Definitions}{31}\fwbtx[
]fwetx=%
\fwcdef 
\fwbeginmacronotes
\fwseealso{This macro is defined in definitions 32 and 33.}
\fwusedin{This macro is invoked in definition 35.}
\fwendmacronotes
\fwendmacro


The following code installs the minor mode and runs any user-provided
hook function.

\fwbeginmacro
\fwmacroname{switch-mode Constructed Code}{33}\fwplusequals \fwodef \fwbtx[(setq minor-mode-alist
      (reverse
       (append (reverse minor-mode-alist)
               (list (list
                      'switch-mode
                      " Switch")))))
(force-mode-line-update t)

(add-hook 'post-command-hook
          'switch-post-command-hook-function)

(run-hooks 'switch-mode-hook)]fwetx=%
\fwcdef 
\fwbeginmacronotes
\fwseealso{This macro is defined in definitions 32 and 33.}
\fwusedin{This macro is invoked in definition 35.}
\fwendmacronotes
\fwendmacro


\fwbeginmacro
\fwmacroname{Standard Header}{34}\fwmany{}\fwequals \fwodef \fwbtx[; A.B.Coates
; Department of Physics
; The University of Queensland  QLD  4072
; Australia
; Standard GNU copyleft provisions apply to this file.]fwetx=%
\fwcdef 
\fwbeginmacronotes
\fwusedin{This macro is invoked in definitions 35 and 39.}
\fwendmacronotes
\fwendmacro


\fwbeginmacro
\fwfilename{switch-mode.el}{35}\fwequals \fwodef \fwbtx[
]fwetx=%
\fwmacroname{Standard Header}{34}\fwbtx[

]fwetx=%
\fwmacroname{switch-mode Constructed Code}{32}\fwbtx[
]fwetx=%
\fwcdef 
\fwbeginmacronotes
\fwisafile{This macro is attached to an output file.}
\fwendmacronotes
\fwendmacro


The file ``{\tt fw-mode.el}'' requires the switch minor mode file
``{\tt switch-mode.el}'', and provides the new mode function
`{\tt fw-mode}'.  Calling the selected \TeX-mode is a way of making
sure that the required definitions are loaded, since the
mode-selection function may be defined as an {\em autoload} function.

\fwbeginmacro
\fwmacroname{fw-mode Constructed Code}{36}\fwplusequals \fwodef \fwbtx[(require 'switch-mode)

(provide 'fw-mode)

]fwetx=%
\fwmacroname{fw-mode Variable Definitions1}{10}\fwbtx[

(funcall fw-TeX-mode)

]fwetx=%
\fwmacroname{fw-mode Variable Definitions2}{8}\fwbtx[
]fwetx=%
\fwcdef 
\fwbeginmacronotes
\fwseealso{This macro is defined in definitions 36, 37 and 38.}
\fwusedin{This macro is invoked in definition 39.}
\fwendmacronotes
\fwendmacro


\noindent In case the user forgets the precise name, some synonyms for
`{\tt fw-mode}' are provided.

\fwbeginmacro
\fwmacroname{fw-mode Constructed Code}{37}\fwplusequals \fwodef \fwbtx[(fset 'FW-mode 'fw-mode)
(fset 'funnelweb-mode 'fw-mode)
(fset 'FunnelWeb-mode 'fw-mode)

]fwetx=%
\fwmacroname{fw-mode Function Definitions}{1}\fwcdef 
\fwbeginmacronotes
\fwseealso{This macro is defined in definitions 36, 37 and 38.}
\fwusedin{This macro is invoked in definition 39.}
\fwendmacronotes
\fwendmacro


\noindent If the keymap has not already been set in the user's
``{\tt .emacs}'' file, then a TeX-mode keymap is copied and
used if possible.

\fwbeginmacro
\fwmacroname{fw-mode Constructed Code}{38}\fwplusequals \fwodef \fwbtx[(if fw-mode-map
 (progn
  (defvar tex-mode-map nil)
  (if tex-mode-map
      (setq fw-mode-map tex-mode-map)
    (progn
      (defvar latex-mode-map nil)
      (if latex-mode-map
          (setq fw-mode-map latex-mode-map)
        (progn
          (defvar LaTeX-mode-map nil)
          (if LaTeX-mode-map
              (setq fw-mode-map LaTeX-mode-map)
            (progn
              (defvar TeX-mode-map nil)
              (if TeX-mode-map
                  (setq fw-mode-map TeX-mode-map))))))))))

]fwetx=%
\fwmacroname{fw-mode Key Definitions}{12}\fwcdef 
\fwbeginmacronotes
\fwseealso{This macro is defined in definitions 36, 37 and 38.}
\fwusedin{This macro is invoked in definition 39.}
\fwendmacronotes
\fwendmacro


\fwbeginmacro
\fwfilename{fw-mode.el}{39}\fwequals \fwodef \fwbtx[
]fwetx=%
\fwmacroname{Standard Header}{34}\fwbtx[

]fwetx=%
\fwmacroname{fw-mode Constructed Code}{36}\fwbtx[
]fwetx=%
\fwcdef 
\fwbeginmacronotes
\fwisafile{This macro is attached to an output file.}
\fwendmacronotes
\fwendmacro


\end{document}

% Local Variables:
% TeX-master: fwmode
% End:


\bye


%*******************************************************************************
%*                    END OF AUTOMATICALLY GENERATED TEX FILE                  *
%*******************************************************************************

